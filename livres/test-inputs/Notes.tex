
\section{Utilisation de GwToLaTeX}

Dans le cadre de mon travail sur la abse généalogique chausiaise, j'ai développé,
au fil des ans, un outil en Python me permettant d'éditer périodiquement une
version papier d'un extrait de cette base.

J'ai récemment (2024) décidé de reprendre ce travail pour en faire un outil
"partageable", en m'efforçant de rendre aussi générique que possible ce que j'avais
décidé de faire pour mon besoin personnel.

La suite de ce document explique comment utiliser cet outil.

\subsection{Principe général}

GwToLaTeX est un outil qui complémente GeneWeb, et présuppose l'existence d'une
base généalogique GeneWeb accessible via le réseau (l'accès en local a amplement
été etsté. l'accès en réseau devrait être possible).
GwToLaTeX fonctionne en effectuant une succession d'appels vers une base GeneWeb,
et en agglomérant en un seul document PDF (appelè "livre") les pages résultant
de ces appels. Attention aux paramètres de votre base GeneWeb qui pourrraient
conduire GeneWeb à considérer GwToLaTeX comme un robot!!
(TODO mettre une temporisation de 1/2 sec dans GwToLaTeX)
Tous les appels vers GeneWeb ne sont pas nécessairement accessibles, en
particulier certains arbres et les tableaux, mais ces restrictions sont
susceptibles d'évoluer.

Les pages actuellement testées sont les pages personnelles
(p=prénom\&n=patronyme\&occ=occ), les listes de descendances (limitées à
une profondeur de 4; m=D\&v=4), les listes par titres (m=TT).

La mise en page de l'ensemble (structuration en chapitres, sections, ...,
numérotation diverses, table des matières, index, références croisées)
est réalisée par l'outil LaTeX.

La liste des pages constituant le "livre" sont définies dans un document source
nommé "famille.txt" et conservé dans un répertoire "livres".
Le document famille.txt peut être agrémenté par des commandes directes au
logiciel LaTeX, et par l'inclusion de fichiers annexes. Ces fichiers annexes
sont conservés dans un sous-répertoire famille-inputs du répertoire livres.

Ainsi, le document que vous lisez a la structure suivante :

\begin{verbatim}
\documentclass[a4paper]{book}

\tracingcommands=1
\tracingparagraphs=1 
\tracingmacros=1
\tracingoutput=1
\makeindex

<x Input ./gw2l_env/Gw2LaTeX-env.tex>

% Edit your own title page data
\begin{titlepage}
\title{Manuel d'utilisation \\
de GwToLaTeX}
\author{Henri Gouraud \\
\texttt{Henri.Gouraud@LaPoste.net} \\
\copyright 2024}
\end{titlepage}

% Beginning of document
\begin{document}

...
Liste de pages et de commandes GwToLaTex
....

% Standard tail of document
\printindex
\newpage
{\huge {\bf Notes}}
\par
Run mkBook du : {\ddmmyyyydate\today} at \currenttime

GeneWeb commit : 
<x Input ./gw2l_dist/gw_version.txt>
\par
GwTo\LaTeX{} version : 
<x Input ./gw2l_dist/version.txt>
\par
<y \hgbaseversion{}
\par
\pdftexbanner{}
\newpage
{\huge {\bf Notes}}
% You may add additional notes pages as needed 
\end{document}
\end{verbatim}

\subsection{Liste des commandes GwToLaTex}

les commandes GwToLaTeX présentes dans le document famille.txt ont la forme
<c Commande paramètres>
Les valeurs de "c" peuvent être:
\begin{description}[style=nextline]
\item[x] Exécute Commande avec paramètre (voir ci-dessous)
\item[y] Ignore la ligne
\item[a] Traite la ligne comme une URL vers GeneWeb. 
Chaque commande "<a" provoque l'ouverture d'une section ou sous-section selon
la valeur du paramètre "BumpSub" ci-dessous. La partie de la ligne comprise entre
">" et "</a>" correspond au titre de la section ou sous-section.
On émet ensuite une commande LaTeX \verb|\index{texte}| avec le texte du titre s'il
n'y a rien après le "</a>", ou le texte entre le "</a>" et la fin de ligne
dans le cas contraire (TODO restaurer ce comportement!!)
Le texte :
\begin{verbatim}
<a href="http://127.0.0.1:2317/%%%BASE%%%?p=leon;n=dupont;oc=0;
    templ=tex;w=%%%PASSWD%%%">Dupont, Léon</a>
\end{verbatim}
donne la page \pageref{section1} et correspond au premier cas, alors que 
\begin{verbatim}
<a href="http://127.0.0.1:2317/%%%BASE%%%?p=lucien;n=dupont;oc=0;
    templ=tex;w=%%%PASSWD%%%">Dupont, Lucien</a>Dupont, Lucien dit Le Frère
\end{verbatim}
correspond au second.
\end{description}

\begin{description}[style=nextline]
\item[Chapter/Section/SubSection/SubSubSection"] Nouveau chapitre/section/...
GwToLatex a besoin de connaitre la numérotation des chapitres et sections afin 
de numéroter correctement les images. Il ne faut donc pas utiliser directement
les commandes LaTeX équivalentes (\verb|\chapter|, \verb|\section|, ...). 
\item["BumpSub"] "on/off". if off, do not increment subsection level.

\item["Input"] Inclure dans le livre le fichier désigné. Dans ce fichier,
remplacer les occurrences de \verb|%%%LIVRES%%%| et \verb|%%%BASE%%%| par
leur valeur telles que définies au lancement du programme 
\item["LaTeX"] Insérer dans le livre une commande LaTeX
\item["Newpage"] Passer à la page suivante
\item["Print"] Imprimer la valeur du paramètre

\item["BumpSub"] "on/off". if off, do not increment subsection level.
usefull is a section has been manually inserted and
is followed by automatic <a ...> (sub)sections
\item["HighLight"] Surligner le nom de la personne (Patronyme prénom (occ))
\item["Hrule"] Ajouter une ligne horizontale de la largeur du teste.

\item["CollectImages"] "on/off", Rassembler les images citées dans les notes
d'une personne pour les imprimer à la fin de la page personnelle.
\item["ImageLabel"] Précise le nombre d'éléments pris en compte dans 
la numérotation des images (ch, sec, ssec, img-nbr)
\item["Sideways"] Imprime la page en mode paysage (utile pour les arbres
qui peuvent être larges).\label{sideways}
\item["Unit"] Définit l'unité (cm, mm, pt)
\item["VignWidth"] Définit la largeur des vignettes (float)
\item["Wide"] "on/off"
\item["ImageWidth"] Définit la largeur des photos (chaine: "textwidth", "5.1cm")
\item["WideImage"] Force la largeur des photos à textwidth
\item["Width"] Définit la largeur des images (float)
\item["NbImgPerLine"] Définit le nombre d'images par ligne (defaut 3) et en
déduit la largeur

\item["Arbres"/"Trees"] "on/off"
\item["FontSize"] Définit la taille de la police de caractères\label{fontsize}
(small, tiny, off returns to default)
\item["TwoPages"] Eclate un arbre sur desu pages (WIP)
\item["TreeMode"] 0 imprime l'arbre comme une liste de liste de cellules (debug),
 1 imprime l'arbre

\item["Version"] Imprime la version du logiciel
\end{description}

\subsection{Les pages GeneWeb}

GeneWeb produit ses résultats à partir de fichiers templates qui peuvent être
adaptés aux besoins des utilisateurs.
GwToLaTeX utilise un jeu de templates conservés dans le dossier tex qui
produisent un mélange de code LaTeX et de pseudo-code.
Les pages GeneWeb qui ont été testées abondamment sont les suivantes :

\begin{description}[style=nextline]
\item[Liste de descendance] Affiche la liste de descendance jusqu'à 4
générations au maximum. Pour faciliter la lecture des descendances plus riches,
la fonction HighLight ci-dessus permet de surligner les personnes
qui font l'objet d'un arbre détaillé (le traditionnel "qui suit" des généalogies).
Les lignes :
\begin{verbatim}
<x Chapter Listes de descendants>
\label{descendance}
<x HighLight Dupont, Léon>
<a href="http://127.0.0.1:2317/%%%BASE%%%?m=D;p=leon+françois;n=dupont;v=4;t=L;
   spouses=on;parents=on;templ=tex;w=hg:%%%PASSWD%%%">Famille Dupont</a>Dupont (Famille)
<a href="http://127.0.0.1:2317/%%%BASE%%%?m=D;p=leon;n=dupont;v=4;t=L;
   spouses=on;parents=on;templ=tex;w=hg:%%%PASSWD%%%">Famille Léon Dupont</a>Dupont, Léon (Famille)
<x HighLight off>
\end{verbatim}
donnent le résultat de la page \pageref{descendance}.

\item[Page perso] Affiche les prénoms complets et alias, puis
un portrait si disponible, juxtaposé avec les
données d'état-civil, naissance/baptême/décès/inhumation et profession
et le couple de parents si connus.

Ces deux pavés sont suivis par la ou les familles avec la liste des enfants.
Dans ces listes, l'utilisation des puces indiquent la présence ($\bullet$)
ou l'absence ($\circ$) de descendance (ou d'ascendance pour les parents).
A noter que ces descendances ou ascendances sont celles mémorisées dans la base
et ne reflètent pas strictement la réalité (les parents par exemple ont toujours
une ascendance dans la réalité!!).

Vient ensuite la note éventuelle apportant des commentaires sur la personne,
complétée par une liste des photos sur laquelle la personne apparaît. 
On reviendra plus loin (page \pageref{images}) sur cette fonctionnalité.
Le dernier pavé cite les sources d'information pour les données collectées


\item[Liste par titre] Obtenues avec la commande
\verb|m=TT;sm=S;t=titre;p=lieu;|

\item[Arbres] Obtenues avec les commandes \verb|m=A;t=T| et \verb|m=D;t=T|.
La mise en page automatique de ces arbres est un problème difficile et
nécessite des tâtonnements an ajustant la profondeur, l'orientation "portrait"
ou "paysage" de la page (commande "Sideways" page \pageref{sideways}) et la
taille de la police de caractères (commande "FontSize" page \pageref{fontsize})

\item[Autres pages] GwToLaTeX interprète les balises HTML et s'efforce de
les traduire dans un équivalent Dans la mesure où HTML et LaTeX n'ont pas
les mêmes conventions et modèle de mise en page, une traduction littérale
n'est pas possible, et échoue dans les cas un peu complexe, les tableaux 
en particulier.

\end{description}

\subsection{Index}

L'utilisation d'un outil automatique permet de construire systématiquement
l'index des personnes citées à un titre ou à un autre dans le livre.
GwToLaTex collecte toutes les références aux personnes
détectées dans les url vers ces personnes. Pour les femmes mariées
GwToLaTex produit deux entrées supplémentaires :
\begin{verbatim}
\index{Non de jeune fille, prénon (ep patronyme du mari)}
\index{Patronyme du mari, prénon (née Non de jeune fille)}
\end{verbatim}
et ajoute aussi les lignes :
\begin{verbatim}
\index{Alias, voir Patronyme, Prénom}
\end{verbatim}
si un alias est défini dans la base.

Au delà de cette collecte GwToLaTex permet d'insérer dans le corps des notes
des commandes LaTeX qui produiront une entrée dans l'index sans perturber
l'affichage sur le Web :
\begin{verbatim}
<span style="display:none mode="tex">\index[Aaaa, Bbbb]</span>
\end{verbatim}
permettant d'ajouter dans l'index des personnes citées dans les notes
sans qu'elles soient présentes dans la base.
Le contenu du <span> peut être n'importe quelle commande LaTeX de votre choix.
ATTENTION : pour une raison obscure, il est nécessaire d'encadrer le
contenu de la commande index par des [ ] plutôt que les { } attendus.

\subsection{Images et index des personnes dans les images}

L'une des forces de GeneWeb est de permettre l'insertion d'images ou de 
liens vers des images dans le corps des notes. La version papier d'une base
se doit de reproduire ces images et le fait en rassemblant à la fin de chaque
page personnelle les images qui y ont été appelées.

Sans avoir recours à la reconnaissance des visages, GwToLaTex se propose de
rajouter dans l'index la référence des photos où les personnes référencées
dans l'index apparaissent. Pour ce faire, GwToLaTex s'appuie sur le fichier
"famille-inputs/who-is-where.txt" que l'utilisateur aura préparé.
La structure de ce fichier est la suivante :

\begin{verbatim}
#
Image-id;Page;Description;Nom-de-fichier.jpg
\index{Wicart, Monseigneur}/z
\index{Gardais, Abbé}/1
\index{Marie, Alain}

#Image suivante
\end{verbatim}
\begin{description}[style=nextline]
\item[Image-id] Identifiant unique de l'image
\item[Page] 0 si l'image apparait dans les notes de la base, n si elle apparait
dans l'annexe (voir section \ref{sec:annexe} page \pageref{annexe}) 
\item[Description] Description de l'image
\item[Nom-de-fichier] Nom du fichier dans le répertoire des images de la base
(\tt{bases/mabase/src/images})
\end{description}

L'index de ce document page \pageref{index} donne un exemple de ce traitement.

\subsection{Annexe}
Quand il n'est pas possible d'incorporer toutes els images souhaitées dans les
notes d'une base, il reste la possibilité d'adjoindre au document produit par 
GwToLaTex une annexe qui sera automatiquement ajoutée au fichier .PDF produit.
Cette annexe a sa propre numérotation de pages.

\subsection{La totalité du fichier test.txt}

See \verb|gwtolatex/livres/test.txt|

\subsection{Améliorations}

Idées d'améliorations :
\begin{description}
\item Simplifier, clarifier homogénéiser les commandes
\item Ajouter les images du carrousel
\end{description}

\subsection{Installation, test et exécution}

GwToLaTex est installé à partir de son répertoire GitHub :
\verb|git clone https://github.com/hgouraud/gwtolatex|
suivi par
\verb|make distrib| or \verb|make install|

Ces deux dernières commandes construisent un répertoire \verb|gw2l_dist| qui
contient tous les éléments nécessaires au fonctionnement de GwToLatex.

\verb|./setup-test.sh| organise les données de la configuration de test et
lance l'exécution du test (\verb|./run-test.sh|) qui construit le présent
fichier (dans \verb|gwtolatex/livres|). Cette dernière commande est
suffisante si vous souhaitez relancer le test après une modification
dans les données de test.

A noter que \verb|./setup-test.sh| fait l'hypothèse que le répertoire
\verb|gwtolatex| est "à côté" du répertoire \verb|geneweb| et relance
l'exécution de \verb|gwd| avec les paramètres appropriés.

GwToLaTeX doit être exécuté à partir du répertoire des bases de
GeneWeb\footnote{Cette contrainte, est liée à la sécurité de Geneweb}.
Il convient donc de recopier le dossier \verb|gw2l_dist| dans le
dossier contenant vos bases. Une fois cette copie faite, l'exécution se
fait avec :
\begin{verbatim}
cd dossier_des_bases
./gw2l_dist/mkBook -base ma_base -livres dossier_livres -family famille
\end{verbatim}

Le répertoire \verb|dossier_livres| contient le fichier \verb|famille.txt|
le fichier (optionnel) \verb|famille-annex.pdf| et un dossier
\verb|famille-input| contenant \verb|who-is-where.txt| (optionnel)
et les fichiers éventuels appelés avec la commande \verb|<x Input fichier>|.
Dans ces fichiers, les chaines de caractères \verb|%%%LIVRES%%%|,
\verb|%%%BASE%%%| et \verb|%%%PASSWD%%%| seront remplacées par les valeurs
des paramètres \verb|-livres|, \verb|-base| et \verb|-passwd|.

ATTENTION : la première étape du processus de fabrication du "livre" construit une 
nouvelle version de la base utilisée dans laquelle les données de référence
croisées des images ont été ajoutées à l'aide du fichier \verb|who-is-where.txt|.
Cette création échouera si des divergences de noms (majuscules, accents) existent
entre la base d'origine et ce fichier.

\subsubsection{Paramètres de lancement}

Le logiciel \verb|mkBook| est lancé avec les paramètres suivants :
\begin{description}
\item[-base] nom de la base utilisée
\item[-bases] "."; mkBook est exécuté dans le dossier bases (WIP)
\item[-passwd] mot de passe magicien (user:pwd) si nécessaire
\item[-livres] dossier des "livres"
\item[-family] nom du "livre"
\item[-famille] idem
\item[-o] nom du fichier résultat. Par défaut \verb|livres/family.pdf|
\item[-debug] mode debug
\item[-level] niveau des messages de debug
\item[-v] exécuter pdflatex en mode verbose
\item[-help] liste des paraètres
\end{description}

